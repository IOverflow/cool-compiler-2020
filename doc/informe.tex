\documentclass[a4paper, 12pt]{article}
    % General Document formatting
    \usepackage[margin=0.7in]{geometry}
    \usepackage[parfill]{parskip}
    \usepackage[utf8]{inputenc}
    \usepackage{graphicx}
    \usepackage{amsthm}
    \usepackage{amsmath}
    \usepackage{mathtools}

\newtheorem{mydef}{Definici\'on}
\newtheorem{mytheo}{Teorema}

\begin{document}
\title{Documentacion del Proyecto final de Complementos de Compilación}
\date{Curso 2019-2020}

\author{  
	Adrian Gonzalez Sánchez- C412 - [@adriangs1996]\\
	Eliane Puerta Cabrera- C412 - [@NaniPuerta]\\
	Liset Alfaro Gonzalez- C411 - [@LisetAlfaro]\\
  }

\maketitle

\section{Introducción}
$COOL (Classroom Object-Oriented Language)$ es un lenguaje pequeño, prácticamente de expresiones; pero que aun así, 
mantiene muchas de las características de los lenguajes de programación modernos, 
incluyendo orientación a objetos, tipado estático y manejo automático de memoria. 
La gramatica de Cool utilizada es libre del contexto y es definida en nuestro proyecto como 
una gram\'atica s-atributada, una definici\'on informal de la misma
puede encontrarse en el manual de Cool que aparece en la carpeta doc, precisamente donde se encuentra esta documentación.\\
En este documento quedan reflejadas las principales ideas a seguir por el equipo para 
la implementación de un compilador funcional de este lenguaje, 
pasando por un lenguaje intermedio CIL que facilita su transición a Mips.

\section{Requerimientos para ejecutar el compilador}
Lo primero que se debe hacer ejecutar el archivo Makefile que se encuentra en la carpeta src del proyecto utilizando el siguiente
comando:
\begin{verbatim}
	$ make
\end{verbatim}

una vez termine de ejecutarse, habr\'a ocurrido lo siguiente
\begin{itemize}
	\item Se tendr\'a instalada la librería cloudpickle, requerida para cargar el Parser que posteriormente mencionaremos.
	\item Se crear\'a la carpeta build donde habr\'a un m\'odulo que contiene el binario de nuestro Lexer y nuestro Parser.
\end{itemize}
\section{Arquitectura:}

\subsection*{Módulos:}
\begin{itemize}
	\item \textbf{Tokenizer}: \textit{src/tknizer.py}
	\item \textbf{Parser}: El parser no es un m\'odulo, en vez de eso utilizamos nuestro
	propio generador de parsers, el cual se encuentra en \textit{src/parserr/slr.py}. La tabla de parsing
	devuelta es lo que se compila con cloudpickle.
	\item \textbf{COOL AST}: \textit{src/abstract/tree.py}
	\item \textbf{Visitors}: \textit{src/travels/*.py}
	\item \textbf{CIL AST}: \textit{src/cil/nodes.py}
	\item \textbf{MIPS Instruction Set}: \textit{src/mips/*.py}
\end{itemize}

\subsection*{Fases en las que se divide el compilador:}
Para una mejor organización se divide el compilador en varias fases
\begin{itemize}
\item \textbf{Lexer}: En esta fase se realiza un preprocesamiento de la entrada. 
Este preprocesamiento consiste en sustituir los comentarios ,que pueden ser anidados y 
por tanto no admiten una expresi\'on regular para detectarlos, por el caracter $espacio$, 
manteniendo los caracteres fin de línea en el archivo fuente 
(Esto es necesario para detectar un errores en la línea y la columna exacta donde ocurren).\\
El tokenizer que recibe este archivo fuente ya procesado tiene una tabla de expresiones regulares 
donde se encuentran clasificados los tokens por prioridad, 
esto nos permite, por ejemplo, no confundir palabras claves(keywords) con identificadores. 
Este devuelve una lista de $tokens$ y pasamos a la segunda fase.\\

\item \textbf{Parser:} El generador del parser fue hecho por el equipo y la implementación 
puede encontrarse en los archivos cuyo nombre corresponden con esta fase. 
Este devuelve un parser $LALR$, que produce una derivaci\'on extrema derecha sobre una cadena de tokens y luego 
esta se evalúa en una función llamada $evaluate$ que devuelve al $AST de Cool$.\\

\item \textbf{Chequeo Sem\'antico:} En esta fase implementan varios recorridos sobre el AST de COOL, siguiendo 
el patr\'on $visitor$. 
Cada uno recolecta errores de índole sem\'antica y si uno de ellos encuentra un error lanza una excepción, 
siendo este el comportamiento deseado:
\begin{itemize}
	\item Recolector de Tipos(TypeCollector): Este recolecta los tipos y los define en el contexto. En este recorrido se
	detecta si se redefinen las clases built-in, o si las clases son redefinidas.
	
	\item Constructor de Tipos(TypeBuilder): Este Visitor define en cada tipo las funciones y atributos, 
	y maneja la herencia y redefinición de funciones. 
	En este se detectan errores como dependencias circulares, los atributos definidos con tipos inexistentes, 
	se chequean la correctitud de las definiciones de los par\'ametros de los m\'etodos que se redefinen, etc..

	\item Inferencia de tipos(Inferer): En este recorrido se realiza el chequeo de Tipos y
	adem\'as incorporamos inferencia de tipos al lenguage, feature que no est\'a contemplado
	en la definici\'on formal. Este recorrido visualiza el codigo de COOL como una gran expresi\'on
	y realiza un an\'alisi Bottom-Up, partiendo de tipos bien definidos de las expresiones en las hojas
	como enteros, strings, etc. Como paso extra, para inferir los tipos nos basamos en las mismas reglas
	semanticas de tipos, pero agregamos recorridos para resolver los problemas de dependencia que puedan surgir
	entre los elementos a inferir. Por ejemplo es posible que para inferir el valor de retorno de un m\'etodo, primero
	haya que inferir el valor de retorno de sus par\'ametros. Para esto se agrega un par\'ametro \textbf{deep} al visitor
	que determina la cantidad de veces que se recorre el AST, y por tanto, determina la profundidad de las dependencias
	que podemos resolver. Para marcar un atributo, par\'ametro, variable o m\'etodo para que el sistema de 
	inferencia intente inferir su tipo, basta con marcar su tipo como \textbf{AUTO\_TYPE}. En caso de que no se
	pueda inferir alg\'un tipo, entonces se lanza un error sem\'antico. 
	
\end{itemize}
\item \textbf{Generación de código:} En esta fase se pasa de Cool a CIL y luego de CIL a MIPS. 
El conjunto de instrucciones de MIPS que se utiliza son del emulador de SPIM, o sea, utilizamos el conjunto
de instrucciones extendidas de MIPS, donde existen instrucciones como LW y MOVE que no son parte de la Arquitectura
original y que se traducen en dos o m\'as instrucciones en modo \textit{bare}. En esta fase primeramente traducimos
el AST de Cool a un AST de CIL (lenguaje intermedio), con el objetivo de facilitar el paso hacia MIPS. Este es un c\'odigo
de tres direcciones, compuesto por tres secciones principales \textbf{.DATA} \textbf{.CODE} \textbf{.TYPES}.
En la secci\'on .DATA almacenamos nuestros strings literales y lo que nosotros llamamos \textit{TDT}. En la secci\'on
de tipos va una representaci\'on de los tipos que luego se puedan convertir f\'acilmente a MIPS, aqu\'i ocurre el mangling
de los nombres de las funciones de modo que se puedan referenciar distintamente luego. Por \'ultimo en la secci\'on
.CODE van todas las funciones que nuestro programa necesite, incluidas las built-in que son bootstrapeadas al
inicio de este recorrido. Es interesante resaltar en CIL como manejamos los m\'etodos y atributos heredados, ya
que en los records que representan los tipos, cada m\'etodo y atributo heredado es agregado primero antes que los definidos
por el tipo en particular; lo que garantiza que para cualquier m\'etodo "$x$", el \'indice de $x$ en la lista de m\'etodos
de cualquier tipo que lo herede, es el mismo. Una vez que se termina el recorrido por el AST de COOL, hemos conformado
las tres secciones de CIL, y estas son pasadas al Generador de MIPS, que no es m\'as que otro recorrido sobre este nuevo
AST que hemos obtenido.

\item \textbf{Generaci\'on de c\'odigo MIPS}: El primer reto para generar c\'odigo ensamblador es representar el
conjunto de instrucciones de la arquitectura. Para lograr esto definimos varias clases bases (branches, loads, stores, comparisson, etc)
que redefinen el m\'etodo \textbf{\_\_str\_\_}, de modo que utilizan el nombre de la clase para representarse, y agregan el s\'imbolo
\$ en dependencia de si ponemos una constante o un registro. De esta forma la generaci\'on de c\'odigo referente a un nodo de CIL
se realiza de manera natural, casi como si program\'aramos en ensamblador, como por ejemplo, en este caso:

\begin{verbatim}
	@visit.register
    def _(self, node: cil.ArgNode):
        # Pasar los argumentos en la pila.
        self.add_source_line_comment(node)
        src = self.visit(node.name)
        reg = self.get_available_register()
        assert src is not None and reg is not None

        if isinstance(src, int):
            self.register_instruction(LI(reg, src))
        else:
            self.register_instruction(LW(reg, src))

        self.comment("Push arg into stack")
        self.register_instruction(SUBU(sp, sp, 4, True))
        self.register_instruction(SW(reg, "0($sp)"))

        self.used_registers[reg] = False
\end{verbatim}

Como se puede ver, registramos una instruccion, casi como si program\'aramos assembly.

\paragraph{}
Otra aspecto significativo es la representaci\'on de los tipos en runtime. Al final, todo son solo n\'umeros,
por lo que tenemos que darles un formato que sea adecuado para poderlos referenciar y, que, adem\'as, permita
acceder a sus propiedades de manera eficiente. En este aspecto, usamos la siguiente estructura para los tipos:

\begin{verbatim}
		     #################################  address
    	#          TYPE_POINTER         #
    	#################################  address + 4
    	#          VTABLE_POINTER       #
    	#################################  address  + 8
    	#           ATTRIBUTE_1         #
    	#################################  address + 12
    	#           ATTRIBUTE_2         #
    	#################################
    	#               ...             #
    	#               ...             #
    	#               ...             #
    	#################################
\end{verbatim}

Esta estructura permite varias cosas:
\begin{itemize}
	\item Acceso a atributos en $O(1)$.
	\item Acceso al nombre del tipo que corresponda a una instancia en $O(1)$.
	\item Dispatch din\'amico en $O(1)$.
\end{itemize}

El mencionado puntero a la VTABLE, no es m\'as que el inicio de un array (Idea sacada de $C++$) que en cada
tipo contiene los nombres manglados de sus funciones, y recordemos que anteriormente dijimos que en CIL, 
garantiz\'abamos que el ind\'ice de cada m\'etodo es el mismo en cada tipo que lo herede, de ah\'i que 
acceder a una funci\'on determinada sobre una instancia es tan f\'acil como cargar el puntero a la VTABLE
de la instancia, calcular el \'indice del m\'etodo que se quiere en tiempo de compilaci\'on y indexar en la 
VTABLE en ese indice en runtime, para un dispatch en $O(1)$.

Otro aspecto que le prestamos inter\'es en esta fase es el tema de los \textit{case}. Las expresiones "case"
requieren que la jerarqu\'ia de tipos sea mantenida de alguna forma en runtime. En vez de esto, calculamos 
lo que llamamos una $TDT$ (Type Distance Table), que no es m\'as que una tabla donde para cada par de tipos
A y B se le asigna la distancia a la que se encuentran en la jerarqu\'ia, o $-1$ en caso de que no pertenezcan
a la misma rama. Para conformar esta tabla hacemos un preprocesamiento en $O(n^2)$ basado en los resultados 
de los tiempos de descubrimiento de un recorrido DFS sobre la jerarqu\'ia de tipos, partiendo desde la ra\'iz,
Object.

Importante tambi\'en son las convenciones que adoptamos al generar c\'odigo. Por ejemplo, cada funci\'on
de MIPS tiene un marco de pila que es creado con un m\'inimo de 32 bytes, porque es convenio entre programadores
de MIPS. Los par\'ametros a funciones se pasan en la pila, ya que aunque MIPS cuenta con 32 registros, no todos
son de prop\'osito general y complica mucho la implementaci\'on el hecho de llevar una cuenta de los registros
utilizados.

\end{itemize}
\end{document}
